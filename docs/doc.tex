\documentclass[11pt,a4paper]{article}
\usepackage[left=2cm,text={17cm,25cm},top=2.5cm]{geometry}
\usepackage[T1]{fontenc}
\usepackage[english]{babel}
\usepackage[utf8]{inputenc}
\usepackage{url}
\usepackage{graphicx}
\usepackage{pdfpages}
\usepackage[colorinlistoftodos,prependcaption,textsize=tiny]{todonotes}

\graphicspath{ {figs/} }

\begin{document}

\begin{center}
	\LARGE{SIN -- Inteligentní systémy}\\
	\Large{Implementace meteostanice}
	\vspace{0.5cm}

    \begin{centering}
    \small{
        Bc. Petr Stehlík <xstehl14@stud.fit.vutbr.cz>\\Bc. Matej Vido <xvidom00@stud.fit.vutbr.cz>
        }
    \end{centering}

	\vspace{0.2cm}

\end{center}

\section{Úvod}
Cílem projektu je navrhnout a implementovat meteostanici využívající MQTT protokol pro zasílání naměřených hodnot serveru. Server implementuje databázi a webové grafické uživatelské rozhraní (GUI), analyzuje historické hodnoty a na jejich základě řídí akce aktuátorů.

Meteostanice měří teplotu, vlhkost a tlak vzduchu; vlhkost v květináči a světelnou intenzitu. Navržené aktuátory jsou ovládání žaluzií, klimatizace, topení a zavlažování rostlin. Vše je autonomně ovládáno na základě předem stanovených pravidel.

GUI zobrazuje aktuální a historické naměřené hodnoty jednotlivých senzorů, trendy a naposledy vykonané akce aktuátorů společně s krátkodobou předpovědí počasí získanou z volně dostupných zdrojů.

\section{Nástroje pro monitoring a řízení}
Grafana (grafy)
Icinga, Nagios
BeeeOn (https://beeeon.org/wiki/Main\_Page)

\section{Architektura a implementace}
\subsection{HW}
zapojení senzorů + názvy, co měří, jak měří
schéma HW zapojení (rpis)
aktuátory

\subsection{SW}
aktuátory
mqtt (co to je, jak se pouziva) + mosquitto + topic schemes
zapojení dashboardu (fe, be, db, rest api)
popis technologii (angular, flask, sqlite)

\section{Výsledky}
\label{sec:res}

\section{Závěr}
\label{sec:sum}

\end{document}
